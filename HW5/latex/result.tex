\documentclass[UTF8]{ctexart}
\usepackage{graphicx}
\usepackage{amsmath}
\pagestyle{plain}   
% \usepackage{booktabs}
% \usepackage{subfigure}
\usepackage{setspace}
\date{}
\title{AVL树与Splay树查找时间对比实验报告} 
\author{杨景凯}
\date{2021/3/22} 
\begin{document} 
% \maketitle 
\begin{center}
    \quad \\
    \quad \\
    \quad \\
    \vskip 3.5cm
    \heiti \zihao{1}AVL树与Splay树查找时间对比实验报告\\
\end{center}
\vskip 3.5cm
\begin{quotation}
    \songti \fontsize{30}{30}
    \doublespacing
    \par\setlength\parindent{12em}
    \quad 
\begin{center}
    %学\hspace{0.61cm} 院:\underline{电子信息与电气工程学院}

    学生姓名:\underline{\qquad    \quad \quad 杨景凯    \quad  \quad\qquad }

    学\hspace{0.61cm} 号:\underline{\quad \quad\quad520021910550\quad\quad}

\end{center}
    
    \centering
    2022年3月22日
\end{quotation}
\clearpage
\tableofcontents
\clearpage
\section{实验介绍}
\subsection{实验数据}

实验需要首先对AVL树和Splay树插入数据,并从插入的数据中寻找一个数据子集作为查找集,对查找集中的数据随机进行一定次数的访问。

设插入的数据量为 n, 要对数据结构查找次数为m,其中查找集的数据量为 k。

在本次实验中,我选取的插入的数据量n为200(0至199),查找次数m为20000,查找集数据k为从1开始的分别占总数据n比例为5\%,25\%,45\%,65\%,85\%的值。即10,50,90,130,170。

\subsection{实验方法}

首先先将数据0至199由随机排列的方式插入AVL树和Splay树。再分别对AVL树和Splay树做m次查找,每次查找为查找从1开始的k个元素。分别统计AVL树和Splay树进行所有查找所需的时间。

\clearpage

\section{实验结果}

实验结果如下表所示:

\begin{center}
    \begin{tabular}{|c|c|c|c|c|c|}
        \hline
        查找比例&	5\%&	25\%&	45\%&	65\%&	85\%\\
        \hline
        AVLTree&	0.018014&	0.066067&	0.110361&	0.164077&	0.217001\\
        \hline
        SplayTree&	0.017209&	0.112802&	0.215909&	0.307465&	0.394576\\
        \hline
    \end{tabular}
\end{center}

\section{实验结论}

\subsection{结果分析}

通过实验结果可以发现,在查找数据量很小,查找次数很多的情况下(即$k \ll n \ll m$时),Splay树的查找时间要短于AVL树,但是其他情况下均为AVL树的查找时间短于Splay树。

随着k增加,Splay树相比于AVL树优势逐渐减小最终变为劣势,且劣势逐渐增加,AVL树优势越来越明显。

因此,在查找数据量很小,查找次数很多的情况下(即$k \ll n \ll m$时),应该首先选择使用Splay树,在其余情况下,应该首先选择使用AVL树。

\subsection{理论分析}

上述结果的产生,是由于AVL树和Splay树的特性决定的。

AVL是自平衡树,在进行搜索时,对所有数据的搜索时间是接近的。

但是Splay树不是自平衡树,它会将近期搜索过的数据放于离根节点较近的位置,缩短再次对该数据搜索的时间。在进行搜索时,不同数据的搜索时间存在较大的差异。如果搜索的数据很久都没有搜索过,那么搜索时间会比较长。但是如果搜索的数据近期被搜索过,那么搜索时间会比较短。

因此,针对于频繁对少量数据的搜索,Splay树体现了对AVL树的优势。
\end{document}
