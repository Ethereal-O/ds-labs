\documentclass[UTF8]{ctexart}
\usepackage{graphicx}
\usepackage{amsmath}
\pagestyle{plain}   
% \usepackage{booktabs}
% \usepackage{subfigure}
\usepackage{setspace}
\date{}
\title{RBTree与AVLTree的对比实验报告} 
\author{杨景凯}
\date{2021/4/5} 
\begin{document} 
% \maketitle 
\begin{center}
    \quad \\
    \quad \\
    \quad \\
    \vskip 3.5cm
    \heiti \zihao{1}RBTree与AVLTree的对比实验报告\\
\end{center}
\vskip 3.5cm
\begin{quotation}
    \songti \fontsize{30}{30}
    \doublespacing
    \par\setlength\parindent{12em}
    \quad 
\begin{center}
    %学\hspace{0.61cm} 院:\underline{电子信息与电气工程学院}

    学生姓名:\underline{\qquad    \quad \quad 杨景凯    \quad  \quad\qquad }

    学\hspace{0.61cm} 号:\underline{\quad \quad\quad520021910550\quad\quad}

\end{center}
    
    \centering
    2022年4月5日
\end{quotation}
\clearpage
\tableofcontents
\clearpage
\section{实验介绍}
\subsection{实验数据}

本次实验数据分为两种,一种为顺序数据,一种为乱序数据。两种数据均为0至20,50,100,200,500,1000的数据。只是一种为顺序排列,一种为乱序排列。

两种数据均提前写入文件,通过程序datacreater生成。主程序运行时,直接调用文件读取数据。这样保证了数据的稳定性,降低了数据改变的难易程度。
\subsection{实验过程}
实验分成两部分,两部分如下:
第一部分,测试在红黑树和 AVL 树下,每组输入集使用顺序和乱序两种方式插入时的耗时和旋转操作次数。
第二部分,测试在红黑树和 AVL 树下,每组输入集使用顺序和乱序两种方式查找时的耗时
\section{实验结果}
\subsection{实验表格}
\begin{center}
    \begin{tabular}{|c|c|c|c|c|c|c|}
        \hline
        测试集&	20-顺序&	20-乱序&	50-顺序&	50-乱序&	100-顺序&	100-乱序\\
        \hline
        RBTree插入时间&	0.000017&	0.000012&	0.000015&	0.000013&	0.000023&	0.000019\\
        \hline
        AVLTree插入时间&	0.00001&	0.000009&	0.000026&	0.000021&	0.00004&	0.000049\\
        \hline
        RBTree旋转次数&	18&	18&	48&	48&	98&	98\\
        \hline
        AVLTree旋转次数&	12&	12&	38&	38&	85&	85\\
        \hline
        RBTree查找时间&	0.000004&	0.000005&	0.000014&	0.000013&	0.000064&	0.000055\\
        \hline
        AVLTree查找时间&	0.000003&	0.000002&	0.000006&	0.000005&	0.000011&	0.000012\\
        \hline
    \end{tabular}   

    \begin{tabular}{|c|c|c|c|c|c|c|}
        \hline
        测试集&	200-顺序&	200-乱序&	500-顺序&	500-乱序&	1000-顺序&	1000-乱序\\
        \hline
        RBTree插入时间&	0.000045&	0.000036&	0.00012&	0.000074&	0.000163&	0.000214\\
        \hline
        AVLTree插入时间&	0.000106&	0.000104&	0.000268&	0.000287&	0.000498&	0.000596\\
        \hline
        RBTree旋转次数&	198&	198&	498&	498&	998&	998\\
        \hline
        AVLTree旋转次数&	182&	182&	479&	479&	976&	976\\
        \hline
        RBTree查找时间&	0.000196&	0.0002&	0.001337&	0.001277&	0.004831&	0.006046\\
        \hline
        AVLTree查找时间&	0.000024&	0.000024&	0.000067&	0.00006&	0.000142&	0.000113\\
        \hline
    \end{tabular} 
\end{center}

\subsection{数据分析}
通过分析发现,大多数情况下(除去在数据量极小时,即只有20个的情况下),RBTree插入时间均短于AVLTree插入时间,RBTree查找时间均长于AVLTree查找时间,但RBTree旋转次数均多于AVLTree旋转次数。

\section{实验分析}
\subsection{理论对比}
理论上,RBTree插入时间应该短于AVLTree插入时间,RBTree查找时间应该长于AVLTree查找时间,RBTree旋转次数应该小于AVLTree旋转次数。
理论上,在数据量极小时(只有20个的情况下),由于不稳定,可能会造成RBTree插入时间长于AVLTree插入时间。实验中两者差别不大,因此合理。
但是,实验中,RBTree旋转次数均多于AVLTree旋转次数。这是与理论不符的。
通过分析,我发现两者旋转次数是接近的,这很可能是我的数据的问题导致的两者次数与理论不符的问题。同时,查看代码可以发现,两者旋转次数并未有较大差异,而应该是两者选择在哪里旋转产生了差异。
上述两点,我认为是导致结果的原因。
\subsection{个人理解}
RBTree是一种B树的变种,它将B树与BST结合,我认为这种结合有三个优点:
第一个是节点结构更加统一,不存在“超级节点”的情况,这有利于节点的统一管理。
第二点是由于底层为BST,因此可以直接通过对BST类的继承实现,许多算法不需要重构。
第三点是实质为B树,因此可以具有B树的优点,例如有利于对磁盘的访问等。
而AVLTree是BST的延申,它通过旋转维持了树的平衡。
两者区别主要在于RBTree为访问磁盘的结构,而AVLTree是主要用于访问内存的结构。
其次,RBTree对于平衡性要求不高,而AVLTree对平衡性要求很高。
再次,RBTree更适合查找,而AVLTree更适合插入。
\end{document}
